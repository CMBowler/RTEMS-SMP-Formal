\chapter{Event Manager Model}

\section{Introduction}

\section{Key Concepts}

\begin{description}
  \item [Event] 32 distinct event values.
  \item [Task]  Unit of execution flow.
  \item [Task Id]  Task identifier,
   including value \texttt{RTEMS\_SELF}.
  \item [ISR] Interrupt Service Routine, which can send events.
  \item [(Input) Condition]
    Specification of what event-set receiver is waiting for,
    and how long it is willing to wait.
    Can also be used to query for pending events using
    a special ``event-set'' value
     \texttt{RTEMS\_PENDING\_EVENTS},
     or to return and clear pending events with
     \texttt{RTEMS\_ALL\_EVENTS}.
  \item[Option Set]
     Defines receiver options: waits, timeouts,
  \item [E-Lifetime] Posted; Pending; Received.
  \item [Target Task] Task to which events are being sent.
  \item [Blocked] A task in the blocked state.
  \item [Ready] A task ready for execution.
  \item [Errors]
     timeout;
     unsatisfied;
     invalid task id;
     invalid address;
     \dots?
  \item[Preemption]
    Task can be en-/dis-abled for preemption.
    Sending tasks may be pre-empted if the send unblocks a
    higher priority task.
  \item[Global Task]
  \item[Node] Nodes can be global or local.
  \item[Remote Request] Sends to remote global tasks
  are forwarded to their node for (re-)sending.
  \item[Clock Tick]
\end{description}

\section{Behaviour}

\subsection{Types}

\subsubsection{Basic Types}

\begin{eqnarray*}
   e              &\in& Event
\\ i,\sigma       &\in& Id
\\ s \in Status &\defs& Success | InvId | UnSat | InvAddr | Timeout
\\ c \in Cond   &\defs& Pending | AllEvt
\\ x \in Opt    &\defs& Wait | NoWait | All | Any | Default
\\ I              &\in& Interval
\end{eqnarray*}

\subsubsection{API Types}

\begin{eqnarray*}
   E \in Events &\defs& \Set{Event}
\\ C \in InCond &\defs& Events + Cond
\\ p              &\in& EventsPtr
\\ X \in Opts   &\defs& \Set{Opt}
\end{eqnarray*}


\subsubsection{System Types}

\begin{eqnarray*}
   TStatus &\defs& EvtBlk | OthrBlk | Ready | Running | Preempted \dots
\\ TPrio &\defs& \Nat
\\ TNode &\defs& \Nat
\\ node \in TaskNode &\defs& Id \pfun TNode
\\ prio \in TaskPrio &\defs& Id \pfun TPrio
\\ sts \in TaskStatus &\defs& Id \pfun TStatus
\\ pend \in TaskEvents &\defs& Id \pfun Events
\\ cond \in TaskCond &\defs& Id \pfun InCond
\\ opts \in TaskOpts &\defs& Id \pfun Opts
\\ pmpt \in TaskPreempt &\defs& Id \pfun \Bool
\\ sys \in Sys
   &\defs&
   TaskNode
   \times TaskPrio
   \times TaskStatus
\\ && {} \times TaskEvents
   \times TaskCond
   \times TaskOpts
   \times \dots
\\ && (node,prio,sts,pend,cond,opts,pmpt\dots)
\end{eqnarray*}

\subsubsection{System Invariants}

Assuming $sys = (node,prio,sts,pend,cond,opts,pmpt\dots)$:

\begin{eqnarray*}
  \bigcup
    \{
      \dom node
      , \dots ,
      \dom pmpt
    \} &=& \dom node
\end{eqnarray*}

\subsubsection{System Functions}

Nullary function $self$ return a task's own id (\texttt{RTEMS\_SELF}).
\begin{eqnarray*}
   self &:& Id \fun Id
\\ self_\sigma &\defs& \sigma
\end{eqnarray*}

\begin{eqnarray*}
   condsat &:& InCond \times Opts \fun Events \fun \Bool
\\ condsat(Pending,X)E &\defs& \True
\\ condsat(AllEvt,X)E     &\defs& \True
\\ condsat(C,All)E &\defs& C \subseteq E
\\ condsat(C,Any)E &\defs& C \cap E \neq \emptyset
\end{eqnarray*}

\begin{eqnarray*}
   updevts &:& InCond \times Opts \fun Events \fun Events^2
\\ updevts(Pending,X)E &\defs& (E,E)
\\ updevts(AllEvt,X)E     &\defs& ()
\\ updevts(C,All)E &\defs& (E\setminus C,C)
\\ updevts(C,Any)E &\defs& (E\setminus C,C \cap E)
\end{eqnarray*}

\newpage
\subsection{Event Send}

Assuming $i$ is valid,
then call $send_\sigma(i,E)sys$ adds events $E$
to the pending events of $i$.
If this does not result in satisfying the input condition of $i$,
then no change occurs to the status of $i$,
and $send_\sigma$ returns success.
If the input condition of $i$ is satisfied,
and $i$ is waiting on events, then it is unblocked.
If $i$ is not waiting on events, its status is not changed.
If $\sigma$ has lower priority than $i$ and is pre-emptable,
then it is pre-empted.
When $\sigma$ is eventually run, then $send_\sigma$ returns success.
\begin{eqnarray*}
   send &:& Id \fun (Id \times Events) \fun Sys \fun (Sys \times Status)
\\ \lefteqn{send_\sigma(i,E)sys}
\\ &\defs&
      \IF i \notin \dom sts \THEN (sys,InvId)
\\ && \ELSE
\\ && \quad pend(i) := pend(i) \cup E
\\ && \quad \IF condsat(cond(i),opts(i))(pend(i)) \land sts(i) = EvtBlk \THEN
\\ && \qquad sts(i) := Ready
\\ && \qquad \IF pmpt(\sigma) \land prio(i)>prio(\sigma) \THEN sts(\sigma) := OthrBlk
\\ && \quad \Box
\\ && \quad (sys',Success)
\\ && \Box
\end{eqnarray*}

\newpage
\subsection{Event Receive}

Call $receive_\sigma(C,X,I,p)sys$ first checks if the
pending events for $\sigma$ satisfy the input condition $(C,X)$.
If so, those events are removed from the pending events,
and are written into the event set indicated by $p$,
and $receive_\sigma$ returns success.
If the input condition is not satisfied,
its behaviour depends on $X$ and $I$.
If $X$ contains $Wait$, then, depending on $I$,
$\sigma$ either blocks waiting for events,
or for events or a timeout.
If $X$ contains $NoWait$,
then the currently satisfied events are copied to $p$
and $receive_\sigma$ returns unsatisfied.

When a blocked call of $receive_\sigma$ is resumed,
it could be because was waiting for events, timeout,
or something else.

If woken by something else, it continues from where it was suspended.

If woken by events,
it first checks that the task that sent the final set of events
needed to satisfy it,
has lower priority than it, and is pre-emptable.
That (most recent sending) task is then pre-empted.

If woken by a timeout, it acts like $NoWait$ was specified.

For a event/timeout wakeup,
it then proceeds to go back and check the input conditions
as described above.
\begin{eqnarray*}
   receive &:& Id \fun (InCond \times Opts \times Interval \times EventsPtr)
\\ && {} \fun Sys \fun (Sys \times Status)
\\ \lefteqn{receive_\sigma(C,X,I,p)sys}
\\ &\defs& \IF C=Pending \THEN p @= pend(\sigma) ; (sys',Success)
\\    && \ELIF condsat(cond(\sigma),opts(\sigma))(pend(\sigma)) \THEN
\\    && \quad (E',C') = updevts(cond(\sigma),opts(\sigma))(pend(\sigma))
\\    && \quad pend(\sigma) = E' ; p@=C'; (sys',Success)
\\    && \ELSE \dots\mbox{indeed}\dots
\end{eqnarray*}

\textbf{Note:}
\textsf{It seems the best conceptual model is that both send and receive
really just post their parameter data into an underlying
event manager sub-system, which then determines if the sender
should be pre-empted, and if the receiver should be unblocked.
This sub-system may be implemented by a mix of send, receive, ISR,
and scheduler(?) code.}
