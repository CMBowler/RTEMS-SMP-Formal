\subsection{Chapter 5 Scheduling Concepts}

\subsubsection{Introduction}

p42:

The scheduler’s sole purpose is to allocate
the  all important resource of processor time
to the various tasks competing for attention.

\subsubsection{Background}

p43:

Priority based scheduling algorithms will always select
the highest priority task that is ready to run
when allocating the processor to a task.

There are a few common methods of accomplishing the mechanics of this algorithm.
These ways involve a list or chain of tasks in the ready state.

p43--44:

Another mechanism is to maintain a list of FIFOs per priority.
When a task is readied,
it is placed on the rear of the FIFO for its priority.
This method is often used with a bitmap to assist in locating
which FIFOs have ready tasks on them.
This data structure has $O(1)$ insert, extract
and find highest ready run-time complexities.

p44:

A red-black tree may be used for the ready queue with the priority as the key.
This data structure has $O(\log(n))$ insert, extract
and find highest ready run-time complexities
while $n$ is the count of tasks in the ready queue.

RTEMS provides four mechanisms which allow the user to alter the task scheduling decisions:
\begin{itemize}
  \item
    user-selectable task priority level
  \item
    task preemption control
  \item
    task timeslicing control
  \item
    manual round-robin selection
\end{itemize}

The evaluation order for scheduling characteristics is always
priority, preemption mode, and timeslicing.

A lower priority level means higher priority (higher importance).

Note that the preemption setting has no effect
on the manner in which a task is scheduled.
It only applies once a task has control of the processor.

p46:

Tasks in an RTEMS system must always be in
one of the five allowable task states.
These states are: executing, ready, blocked, dormant, and non-existent.

State Diagram:
\begin{eqnarray*}
   NonExistent  &\defs&  create      \then Dormant
\\ Dormant      &\defs&  deleting    \then NonExistent
                         \extc
                         starting    \then Ready
\\ Ready       &\defs&   deleting    \then NonExistent
                         \extc
                         dispatching \then Executing
                         \extc
                         blocking    \then Blocked
\\ Executing   &\defs&   deleting    \then NonExistent
                         \extc
                         yielding    \then Ready
                         \extc
                         blocking    \then Blocked
\\ Blocked     &\defs&   deleting    \then NonExistent
                         \extc
                         readying    \then Ready
\end{eqnarray*}

p47:

A blocked task may also be suspended.
Therefore,
both the suspension and the blocking condition must be removed
before the task becomes ready to run again.

\subsubsection{Uniprocessor Schedulers}

p49:

5.3.1 Deterministic Priority Scheduler
\dots
It schedules tasks using a priority based algorithm
which takes into account preemption.
\dots
an array of FIFOs with a FIFO per priority.
a bitmap which is used to track which priorities have ready tasks.
\dots
deterministic (e.g., predictable and fixed) in execution time.

\textit{More single-core scheduler descriptions follow\dots}

\subsubsection{SMP Schedulers}

p51:

All SMP schedulers included in RTEMS are priority based.

\paragraph{Earliest Deadline First SMP Scheduler}

job-level fixed-priority scheduler using the Earliest Deadline First

supports \emph{task processor affinities}
of one-to-one and one-to-all, e.g., a task can execute on exactly one processor
or all processors managed by the scheduler instance.

The processor affinity set of a task must
contain all online processors to select the one-to-all affinity.

supports \emph{thread pinning}

ready queues use a red-black tree with the task priority as the key.

\paragraph{Deterministic Priority SMP Scheduler}

A fixed-priority scheduler which uses a table of chains
with one chain per priority level for the ready tasks.

\paragraph{Simple Priority SMP Scheduler}

A fixed-priority scheduler which uses a sorted chain for the ready tasks.

\paragraph{Arbitrary Processor Affinity Priority SMP Scheduler}

A fixed-priority scheduler which uses a table of chains with one chain per priority level for the
ready tasks.

This scheduler supports arbitrary task processor affinities
