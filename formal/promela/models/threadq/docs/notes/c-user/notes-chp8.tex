\subsection{Chapter 8 Interrupt Manager}

\subsubsection{Introduction}

p120:

This manager permits quick interrupt response times by
providing the critical ability to alter task execution
which allows a task to be preempted upon exit from an ISR.

\subsubsection{Background}

p121:

The interrupt manager allows the application to connect a function
to a hardware interrupt vector.
When an interrupt occurs,
the processor will automatically vector to RTEMS.
RTEMS saves and restores all registers which are not preserved
by the normal C calling convention for the target processor
and invokes the user’s ISR.
The user’s ISR is responsible for processing the interrupt,
clearing the interrupt if necessary,
and device specific manipulation.

The \verb"rtems_interrupt_catch" directive
connects a procedure to an interrupt vector.
The vector number is managed using the \verb"rtems_vector_number" data type.

The interrupt service routine is assumed to abide by these conventions
and have a prototype similar to the following:

\begin{nicec}
rtems_isr user_isr(
  rtems_vector_number vector
);
\end{nicec}

Once the user’s ISR has completed,
it returns control to the RTEMS interrupt manager
which will perform task dispatching
and restore the registers saved before the ISR was invoked.

A system call made by the ISR may have readied a task of
higher priority than the interrupted task.

No dispatch processing is performed as part of directives
which have been invoked by an ISR.

Applications must adhere to the following rule if proper task scheduling and dispatching is to
be performed:

\textbf{Note:} \textit{The interrupt manager must be used for all ISRs which may be interrupted by the highest
priority ISR which invokes an RTEMS directive.}

Interrupts are nested whenever an interrupt occurs during the execution of another ISR. RTEMS
supports efficient interrupt nesting by allowing the nested ISRs to terminate without performing
any dispatch processing. Only when the outermost ISR terminates will the postponed dispatching
occur.

p122:

During the execution of directive calls,
critical sections of code may be executed.
When these sections are encountered,
RTEMS disables all maskable interrupts
before the execution of the section
and restores them to the previous level upon completion of the section.

Non-maskable interrupts (NMI) cannot be disabled,
and ISRs which execute at this level MUST NEVER issue RTEMS system calls.

However,
ISRs that make no system calls may safely execute as non-maskable interrupts.

\subsubsection{Operations}

p123:

8.3.2 Directives Allowed from an ISR

Directives invoked by an ISR must operate only on objects
which reside on the local node.
