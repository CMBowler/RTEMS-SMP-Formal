\chapter{Event Manager Documentation Summary}

From
\url{https://docs.rtems.org/branches/master/c-user/event/index.html}

\section{Introduction}

Two directives:
\begin{description}
  \item [EVENT\_SEND] Send event set to a task.
  \item [EVENT\_RECEIVE] Receive event condition.
\end{description}

\section{Background}

\subsection{Event Sets}

Event lifetime:
\begin{description}
  \item [Posted] Directed to a task.
  \item [Pending] Posted, but not yet received.
  \item [Condition] Set by receiving task,
     defining set of events it wishes to receive.
  \item [Received] Event was singled out for reception by target task,
    from its set of pending events.
\end{description}
A receiver can either accept an event,
in its condition set,
 once at least one is pending (\texttt{RTEMS\_EVENT\_ANY}),
or wait for all to be pending (\texttt{RTEMS\_EVENT\_ALL}),
before selecting one.


Event flag used by tasks or ISR to inform task of situations.
There are 32 flags, and we have sets (\texttt{rtems\_event\_set}) of these.

Events are:
simple synchronisers;
aimed at tasks;
independent of each other;
do not have data;
are not queued.

Tasks can wait on a set of events.

\subsection{Building an Event Set or Condition}

A 32-bit word is used to represent an event set,
built from events (using bitwise-OR)
$$\{\texttt{RTEMS\_EVENT\_0},\dots,\texttt{RTEMS\_EVENT\_31}\}$$.

\subsection{Building an \texttt{EVENT\_RECEIVE} Option Set}

Valid options, are as follows:
\begin{description}
  \item [\texttt{RTEMS\_WAIT}] Wait for event (default).
  \item [\texttt{RTEMS\_NO\_WAIT}] Do not wait for event.
  \item [\texttt{RTEMS\_EVENT\_ALL}] Return after all events (default).
  \item [\texttt{RTEMS\_EVENT\_ANY}] Return after any events.
\end{description}
First two above are mutually exclusive, as are second two.
Can use bitwise-OR and addition operators to build these.
We can specify all defaults above using \texttt{RTEMS\_DEFAULT\_OPTIONS}.

\newpage
\section{Operations}

\subsection{Sending an Event Set}

Directive \texttt{rtems\_event\_send}
directs an event set to a target task.

The event set is posted and left pending,
except if the target task is blocked on events and its input event condition
is satisfied.
In this case that task is made ready for execution.

\textbf{Comment:}
\textsf{
  Surely the event set also needs to be posted so the unblocked target
  task can return the relevant satisfying event set?
}

\subsection{Receiving an Event Set}

Directive \texttt{rtems\_event\_receive}
is used to accept an input event condition,
and specify when its request is satisfied.
If satisfied, then a success code and event set are returned.
If not, then the directive, based on what was specified,
either:
waits forever;
waits for a designated time before giving a timeout error;
or returns immediately with an error.

\subsection{Determining the Pending Event Set}

A receive call with input condition \texttt{RTEMS\_PENDING\_EVENTS}
will return the pending event set without altering it.

\subsection{Receiving all Pending Events}

A receive call with \texttt{RTEMS\_ALL\_EVENTS} for input condition
and option set $\{\texttt{RTEMS\_NO\_WAIT},\texttt{RTEMS\_EVENT\_ANY}\}$,
will return, and clear, all pending events.
If no such events, then status code \texttt{RTEMS\_UNSATISFIED} is returned.

\newpage
\section{Directives}

\subsection{\texttt{EVENT\_SEND} --- Send event set to a task}

\begin{verbatim}
rtems_status_code rtems_event_send (
    rtems_id         id,
    rtems_event_set  event_in
);
\end{verbatim}

Status codes:
\begin{description}
  \item [\texttt{RTEMS\_SUCCESSFUL}] send succesfull
  \item [\texttt{RTEMS\_INVALID\_ID}]  invalid task id.
\end{description}

Description:
\begin{quote}
``
This directive sends an event set, \texttt{event\_in}, to the task specified by id.
If a blocked task’s input event condition is satisfied by this directive,
then it will be made ready.
If its input event condition is not satisfied,
then the events satisfied are updated
and the events not satisfied are left pending.
If the task specified by id is not blocked waiting for events,
then the events sent are left pending.
``
\end{quote}
\textbf{Comment:}
\textsf{
The nesting of the ``if``s above is not clear.
In particular, are the events satisfied always updated?
}

Notes:
\begin{quote}
``
Specifying \texttt{RTEMS\_SELF} for id results in the event set being sent
to the calling task.
\par
Identical events sent to a task are not queued.
In other words, the second, and subsequent,
posting of an event to a task
before it can perform an \texttt{rtems\_event\_receive} has no effect.
\par
The calling task will be preempted if it has preemption enabled
and a higher priority task is unblocked as the result of this directive.
\par
Sending an event set to a global task
which does not reside on the local node
will generate a request
telling the remote node to send the event set to the appropriate task.
``
\end{quote}

\newpage
\subsection{\texttt{EVENT\_RECEIVE} --- Receive event condition}

\begin{verbatim}
rtems_status_code rtems_event_receive (
    rtems_event_set  event_in,
    rtems_option     option_set,
    rtems_interval   ticks,
    rtems_event_set *event_out
);
\end{verbatim}

Status codes:
\begin{description}
  \item [\texttt{RTEMS\_SUCCESSFUL}] receive succesfull
  \item [\texttt{RTEMS\_UNSATISFIED}]
     input event not satisfied
     (\texttt{RTEMS\_NO\_WAIT})
  \item [\texttt{RTEMS\_INVALID\_ADDRESS}] \texttt{event\_out} is null
    \item [\texttt{RTEMS\_TIMEOUT}] timed out.
\end{description}

Description:
\begin{quote}
This directive attempts to receive the event condition specified in event\_in.
If event\_in is set to RTEMS\_PENDING\_EVENTS,
then the current pending events are returned in event\_out and left pending.
The RTEMS\_WAIT and RTEMS\_NO\_WAIT options in the option\_set parameter
are used to specify whether or not the task is willing
to wait for the event condition to be satisfied.
RTEMS\_EVENT\_ANY and RTEMS\_EVENT\_ALL are used in the option\_set parameter
are used to specify whether a single event
or the complete event set is necessary to satisfy the event condition.
The event\_out parameter is returned to the calling task
with the value that corresponds to the events in event\_in that were satisfied.
\par
If pending events satisfy the event condition,
then event\_out is set to the satisfied events
and the pending events in the event condition are cleared.
If the event condition is not satisfied and RTEMS\_NO\_WAIT is specified,
then event\_out is set to the currently satisfied events.
If the calling task chooses to wait,
then it will block waiting for the event condition.
\par
If the calling task must wait for the event condition to be satisfied,
then the timeout parameter is used to specify the maximum interval to wait.
If it is set to RTEMS\_NO\_TIMEOUT, then the calling task will wait forever.
``
\end{quote}

Notes:
\begin{quote}
``
This directive only affects the events specified in event\_in.
Any pending events that do not correspond to
any of the events specified in event\_in will be left pending.
``
\end{quote}

Receive option constants:
\begin{description}
  \item [\texttt{RTEMS\_WAIT}] wait for event (default)
  \item [\texttt{RTEMS\_NO\_WAIT}] return immediately
  \item [\texttt{RTEMS\_EVENT\_ALL}] return after all events (default)
  \item [\texttt{RTEMS\_EVENT\_ANY}] return after any events
\end{description}
Clock tick is required.
