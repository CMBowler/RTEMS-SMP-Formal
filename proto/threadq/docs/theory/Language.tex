\section{Program Abstract Syntex}

We present here an abstract syntax of the subset of C in use
with the RTEMS-SMP Thread Queue implementation.


\subsection{Basic Syntax}

This is the basic ``While'' language used as a basis for most semantics
work in this area.

\begin{eqnarray*}
   P,Q \in While
   &::=&  x := e       \quad\mbox{--- Assignment}
\\ & | &  P ; Q        \quad\mbox{--- Sequential Composition}
\\ & | &  P \cond c Q  \quad\mbox{--- Conditional Statement}
\\ & | &  c \whl P     \quad\mbox{--- While Loop}
\\ & | &  Skip         \quad\mbox{--- No-op}
\\ & | &  P \ndc Q     \quad\mbox{--- Demonic Non-determinism}
\end{eqnarray*}
The latter two constructs play a useful role in the theory.

\subsubsection{Derived Constructs}

Here we add other common language constructs onto the basic language above,
and show how they map to those above.

\begin{eqnarray*}
   While &::=& \dots
\\ & | & P \dowhile c   \quad\mbox{--- Do-While loop}
\end{eqnarray*}


\begin{eqnarray*}
    P \dowhile c &\defs&  P ; c \whl P
\end{eqnarray*}

A common idiom used
when a CPP macro is producing  block-structured code $P$,
is to generate $P \dowhile false$.
This is done as a workaround for the limitations of the C preprocessor,
and not because there is some mysterious difference in its behaviour,
as compared to just $P$.
So we will always simplify such usage when modelling.

$$ P \dowhile false \equiv P$$


\newpage
\subsection{Modelling Atomicity}

The ``While'' language is too coarse.
We need to break expression evaluation down, and separate out loads and stores.

\begin{eqnarray*}
   && x:=e
\\ &\equiv& x := f(v_1,\dots,v_n)
\\ &\equiv& load(v_1);\dots;load(v_n);apply(f);store(x)
\\
\\ && P \cond c Q
\\ &\equiv& P \cond{f(v_1,\dots,v_n)} Q
\\ &\equiv& load(v_1);\dots;load(v_n);c_0:=apply(f);P \cond{c_0} Q
\\
\\ && c \whl P
\\ &\equiv& P; c \whl P \cond{c} Skip
\\ &\equiv& P; c \whl P \cond{f(v_1,\dots,v_n)} Skip
\\ &\equiv& load(v_1);\dots;load(v_n);c_0:=apply(f);(P; c \whl P) \cond{c_0} Skip
\end{eqnarray*}

We also need to recognise that loads and stores
(other than atomic ones)
are not atomic, but can involve a number of phases.

This strongly supports the idea of using Promela!

\newpage
\subsection{Modelling Pointers}

RTEMS code makes essential use of pointers.
We need an appropriate abstraction of these.

The following key roles have been identified for pointers
in the thread queue implementation:
\begin{itemize}
  \item
    Pointing to relevant RTEMS objects,
    such as thread control blocks, semaphores, or locks
  \item
    Forming the implementation structure of FIFO queues
    (a.k.a. chains).
  \item
    Forming the implementation structure of priority queues
    (a.k.a. red-black trees).
\end{itemize}
The issue is,
do we assume symbolic queues,
or do we model their behaviour down at the pointer manipulation level?

The simplest model is one that imagines memory as a mapping
from location to value. A pointer is simply a location.
\begin{eqnarray*}
   \ell \in Loc &=& \mbox{finite set of memory locations}
\\ d \in Datum &=& \mbox{finite set of data values}
\\ n \in Mem &=& Loc \fun Datum
\end{eqnarray*}
We can define registers, memory and (atomic) load/store as follows:
\begin{eqnarray*}
   a \in Addr &=& Loc \mbox{ ---Available RAM address space}
\\   MEM &=& Addr \fun Datum
\\  ram &:& MEM
\\ r \in Reg  &=& Loc \mbox{ ---Available CPU registers}
\\  REG &=& Reg \fun Datum
\\  reg &:& REG
\\  (reg,ram), state \in State &=& REG \times MEM
\\  load &:& (Reg \times Addr) \fun State \fun State
\\  load[r,a](reg,mem) &\defs& (reg\override\maplet r {ram(a)},ram)
\\  store &:& (Reg \times Addr) \fun State \fun State
\\  store[r,a](reg,mem) &\defs& (reg,ram\override\maplet a {reg(r)})
\end{eqnarray*}
Pointers are now easy:
\begin{eqnarray*}
   p \in Ptr &=& Addr
\\ 0 &:& Ptr \mbox{ ---The null pointer}
\end{eqnarray*}
