\section{Data Types}\label{sec:datatypes}

Here, we describe the various datatypes in use,
mostly C \texttt{struct}s, and C11 atomic base types and values.

We assume the following regarding key CPP macros:
\begin{description}
  \item [defined:]
    \verb"RTEMS_SMP"
    \verb"_RTEMS_SCORE_CPUSTDATOMIC_USE_STDATOMIC"
  \item [not defined:]
    \verb"RTEMS_DEBUG"
    \verb"RTEMS_PROFILING"
    \verb""
\end{description}
All snippets below are as a result of the above settings being applied.

\subsection{C11 Atomics}

\subsubsection{Atomic Order Relaxed}

From \texttt{include/rtems/score/atomic.h:48}:
\begin{nicec}
#define ATOMIC_ORDER_RELAXED CPU_ATOMIC_ORDER_RELAXED
\end{nicec}
From \texttt{include/rtems/score/cpustdatomic.h:96}:
\begin{nicec}
#define CPU_ATOMIC_ORDER_RELAXED memory_order_relaxed
\end{nicec}

\subsubsection{Atomic Uint}

From \texttt{include/rtems/score/atomic.h:38}:
\begin{nicec}
typedef CPU_atomic_Uint Atomic_Uint;
\end{nicec}
From \texttt{include/rtems/score/cpustdatomic.h:86}:
\begin{nicec}
typedef atomic_uint CPU_atomic_Uint;
\end{nicec}

\newpage
\subsection{Chains}

\subsubsection{Chain Node}

From \texttt{include/rtems/score/chain.h:69}:
\begin{nicec}
typedef struct Chain_Node_struct Chain_Node;
struct Chain_Node_struct {
  Chain_Node *next;
  Chain_Node *previous;
};
\end{nicec}

\subsubsection{Chain Control}

From \texttt{include/rtems/score/chain.h:87}:
\begin{nicec}
typedef union {
  struct {
    Chain_Node Node;
    Chain_Node *fill;
  } Head;

  struct {
    Chain_Node *fill;
    Chain_Node Node;
  } Tail;
} Chain_Control;
\end{nicec}

\newpage
\subsection{Locks}

\subsubsection{SMP Ticket Lock Control}

From \texttt{include/rtems/score/smplockticket.h:41-44}:
\begin{nicec}
typedef struct {
  Atomic_Uint next_ticket;
  Atomic_Uint now_serving;
} SMP_ticket_lock_Control;
\end{nicec}
Both initialised to \texttt{0U}.


\subsubsection{ISR Lock Control}

From \texttt{include/rtems/score/isrlock.h:63}:
\begin{nicec}
typedef struct {
  SMP_lock_Control Lock;
} ISR_lock_Control;
\end{nicec}
From \texttt{include/rtems/score/smplock.h:87}:
\begin{nicec}
typedef struct {
  SMP_ticket_lock_Control Ticket_lock;
} SMP_lock_Control;
\end{nicec}

\newpage
\subsection{Thread Queues}

\subsubsection{Thread Queue Gate}

From \texttt{cpukit/include/rtems/score/threadq.h:121}:
\begin{nicec}
typedef struct {
  Chain_Node Node;
  Atomic_Uint go_ahead;
} Thread_queue_Gate;
\end{nicec}

\subsubsection{Thread Queue Heads}

From \texttt{cpukit/include/rtems/score/threadq.h:366}:
\begin{nicec}
typedef struct _Thread_queue_Heads {
  union {
    Chain_Control Fifo;
    // alternate not here if doing RTEMS_SMP
  } Heads;
  Chain_Control Free_chain;
  Chain_Node Free_node;
  Thread_queue_Priority_queue Priority[ RTEMS_ZERO_LENGTH_ARRAY ];
} Thread_queue_Heads;
\end{nicec}


\subsubsection{Thread Queue Queue}

\begin{nicec}
struct Thread_queue_Queue {
  SMP_ticket_lock_Control Lock;
  Thread_queue_Heads *heads;
  Thread_Control *owner;
  const char *name;
};
\end{nicec}

\subsubsection{Thread Queue Lock Context}

\begin{nicec}
typedef struct {
  ISR_lock_Context Lock_context;
  struct {
    Thread_queue_Gate Gate;
    Thread_queue_Queue *queue;
  } Wait;
} Thread_queue_Lock_context;
\end{nicec}

\subsubsection{Thread Queue Control}

\begin{nicec}
typedef struct {
  Thread_queue_Queue Queue; // The actual thread queue.
} Thread_queue_Control;
\end{nicec}



\newpage

\subsection{Threads}

\subsubsection{Thread Wait Information}


\begin{nicec}
typedef struct {
  uint32_t              count;
  void                 *return_argument;
  Thread_Wait_information_Object_argument_type
                        return_argument_second;
  uint32_t              option;
  uint32_t              return_code;
  Atomic_Uint           flags;
  struct {
    ISR_lock_Control Default;
    Chain_Control Pending_requests; // chain of Thread_queue_Gate !!
    Thread_queue_Gate Tranquilizer;
  } Lock;
  Thread_queue_Link Link;
  Thread_queue_Queue *queue;
  const Thread_queue_Operations *operations;
  Thread_queue_Heads *spare_heads;
}  Thread_Wait_information;
\end{nicec}

\newpage
\subsubsection{Thread Control}

We assume the following are undefined:
\\ \verb"RTEMS_SCORE_THREAD_ENABLE_RESOURCE_COUNT"


\begin{nicec}
struct _Thread_Control {
  Objects_Control          Object;
  Thread_queue_Control     Join_queue;
  States_Control           current_state;
  Priority_Node            Real_priority;
  Thread_Scheduler_control Scheduler;
  Thread_Wait_information  Wait;
  Thread_Timer_information Timer;
     /*================= end of common block =================*/
  bool                                  is_idle;
  bool                                  is_preemptible;
  bool                                  is_fp;
  bool was_created_with_inherited_scheduler;
  uint32_t                              cpu_time_budget;
  Thread_CPU_budget_algorithms          budget_algorithm;
  Thread_CPU_budget_algorithm_callout   budget_callout;
  Timestamp_Control                     cpu_time_used;
  Thread_Start_information              Start;
  Thread_Action_control                 Post_switch_actions;
  Context_Control                       Registers;
#if ( CPU_HARDWARE_FP == TRUE ) || ( CPU_SOFTWARE_FP == TRUE )
  Context_Control_fp                    *fp_context;
#endif
  struct _reent                         *libc_reent;
  void                                  *API_Extensions[ THREAD_API_LAST + 1 ];
  Thread_Keys_information               Keys;
  Thread_Life_control                   Life;
  Thread_Capture_control                Capture;
  struct rtems_user_env_t               *user_environment;
  struct _pthread_cleanup_context       *last_cleanup_context;
  struct User_extensions_Iterator       *last_user_extensions_iterator;
  void                                  *extensions[ RTEMS_ZERO_LENGTH_ARRAY ];
};
\end{nicec}

\verb"Thread_Control" is a typedef of the above.


\newpage
\subsection{Priority}

\subsubsection{Priority Levels}

From \texttt{include/rtems/score/priority.h:72}:
\begin{nicec}
 // Lower values represent higher priorities.  So, a priority value of zero
 // represents the highest priority thread.  This value is reserved for internal
 // threads and the priority ceiling protocol.
typedef uint64_t Priority_Control;

#define PRIORITY_MINIMUM      0
#define PRIORITY_DEFAULT_MAXIMUM      255 // or higher, arch dependent
\end{nicec}

\subsubsection{Priority Node}

From \texttt{include/rtems/score/priority.h:100}:
\begin{nicec}
typedef struct {
  union {
    Chain_Node Chain;
    RBTree_Node RBTree;
  } Node;
  Priority_Control priority;
} Priority_Node;
\end{nicec}

\subsubsection{Priority Aggregation}

From \texttt{include/rtems/score/priority.h:118}:
\begin{nicec}
typedef enum {
  PRIORITY_ACTION_ADD,
  PRIORITY_ACTION_CHANGE,
  PRIORITY_ACTION_REMOVE,
  PRIORITY_ACTION_INVALID
} Priority_Action_type;

typedef struct Priority_Aggregation Priority_Aggregation;

struct Priority_Aggregation {
  Priority_Node Node; // Overall priority
  RBTree_Control Contributors;
  const struct _Scheduler_Control *scheduler;
  struct {
    Priority_Aggregation *next;
    Priority_Node *node;
    Priority_Action_type type;
  } Action;
};
\end{nicec}

\subsubsection{PriorityActions}

From \texttt{include/rtems/score/priority.h:195}:
\begin{nicec}
typedef struct {
  Priority_Aggregation *actions;
} Priority_Actions;
\end{nicec}

\newpage
\subsection{Scheduler}

\subsubsection{Scheduler Operations}

A struct with pointers to scheduler function implementations.
From \texttt{include/rtems/score/scheduler.h:38}:
\begin{nicec}
typedef struct {
  void ( *initialize )( const Scheduler_Control * );
  void ( *schedule )( const Scheduler_Control *, Thread_Control *);
  void ( *yield )(
    const Scheduler_Control *,
    Thread_Control *,
    Scheduler_Node *
  );
  void ( *block )(
    const Scheduler_Control *,
    Thread_Control *,
    Scheduler_Node *
  );
  void ( *unblock )(
    const Scheduler_Control *,
    Thread_Control *,
    Scheduler_Node *
  );
  void ( *update_priority )(
    const Scheduler_Control *,
    Thread_Control *,
    Scheduler_Node *
  );
  Priority_Control ( *map_priority )(
    const Scheduler_Control *,
    Priority_Control
  );
  Priority_Control ( *unmap_priority )(
    const Scheduler_Control *,
    Priority_Control
  );
  bool ( *ask_for_help )(
    const Scheduler_Control *scheduler,
    Thread_Control          *the_thread,
    Scheduler_Node          *node
  );
  void ( *reconsider_help_request )(
    const Scheduler_Control *scheduler,
    Thread_Control          *the_thread,
    Scheduler_Node          *node
  );
  void ( *withdraw_node )(
    const Scheduler_Control *scheduler,
    Thread_Control          *the_thread,
    Scheduler_Node          *node,
    Thread_Scheduler_state   next_state
  );
  void ( *pin )(
    const Scheduler_Control *scheduler,
    Thread_Control          *the_thread,
    Scheduler_Node          *node,
    struct Per_CPU_Control  *cpu
  );
  void ( *unpin )(
    const Scheduler_Control *scheduler,
    Thread_Control          *the_thread,
    Scheduler_Node          *node,
    struct Per_CPU_Control  *cpu
  );
  void ( *add_processor )(
    const Scheduler_Control *scheduler,
    Thread_Control          *idle
  );
  Thread_Control *( *remove_processor )(
    const Scheduler_Control *scheduler,
    struct Per_CPU_Control  *cpu
  );
  void ( *node_initialize )(
    const Scheduler_Control *,
    Scheduler_Node *,
    Thread_Control *,
    Priority_Control
  );
  void ( *node_destroy )( const Scheduler_Control *, Scheduler_Node * );
  void ( *release_job ) (
    const Scheduler_Control *,
    Thread_Control *,
    Priority_Node *,
    uint64_t,
    Thread_queue_Context *
  );
  void ( *cancel_job ) (
    const Scheduler_Control *,
    Thread_Control *,
    Priority_Node *,
    Thread_queue_Context *
  );
  void ( *tick )( const Scheduler_Control *, Thread_Control * );
  void ( *start_idle )(
    const Scheduler_Control *,
    Thread_Control *,
    struct Per_CPU_Control *
  );
  Status_Control ( *set_affinity )(
    const Scheduler_Control *,
    Thread_Control *,
    Scheduler_Node *,
    const Processor_mask *
  );
} Scheduler_Operations;
\end{nicec}

\subsubsection{Scheduler Context}

From \texttt{include/rtems/score/scheduler.h:247}:
\begin{nicec}
typedef struct Scheduler_Context {
  ISR_LOCK_MEMBER( Lock )
  Processor_mask Processors;
} Scheduler_Context;
\end{nicec}

\subsubsection{Scheduler Control}

From \texttt{include/rtems/score/scheduler.h:264}:
\begin{nicec}
struct _Scheduler_Control {
  Scheduler_Context *context;
  Scheduler_Operations Operations;
  Priority_Control maximum_priority;
  uint32_t name;
  bool is_non_preempt_mode_supported;
};
\end{nicec}

\subsubsection{Scheduler Node}

From \texttt{include/rtems/score/schedulernode.h:80}
\begin{nicec}
struct Scheduler_Node {
  union {
    Chain_Node Chain;
    RBTree_Node RBTree;
  } Node;
  int sticky_level;
  struct _Thread_Control *user;
  struct _Thread_Control *idle;
  struct _Thread_Control *owner;
  struct {
    Chain_Node Wait_node;
    union {
      Chain_Node Chain;
      Scheduler_Node *next;
    } Scheduler_node;
    Scheduler_Node *next_request;
    Scheduler_Node_request request;
  } Thread;
  struct {
    Priority_Aggregation Priority;
  } Wait;
  struct {
    Priority_Control value;
    SMP_sequence_lock_Control Lock;
  } Priority;
};
\end{nicec}



\newpage
\subsection{MrsP Control}

\begin{nicec}
typedef struct {
  Thread_queue_Control Wait_queue; // manage ownership and waiting threads
  Priority_Node Ceiling_priority;  // used by the owner thread.
  // One ceiling priority per scheduler instance:
  Priority_Control ceiling_priorities[ RTEMS_ZERO_LENGTH_ARRAY ];
} MRSP_Control;
\end{nicec}

\newpage
\subsection{Semaphore Control}

\begin{nicec}
typedef struct {
  Objects_Control          Object;
  union {  //-- only interested in TQ & MRSP variants
    Thread_queue_Control Wait_queue;
    // CORE_ceiling_mutex_Control Mutex;
    // CORE_semaphore_Control Semaphore;
    MRSP_Control MRSP;
  } Core_control;
}   Semaphore_Control;
\end{nicec}
