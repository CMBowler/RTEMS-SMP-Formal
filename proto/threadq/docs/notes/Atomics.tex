\section{C11 Atomics}

\subsection{Introduction}

The code uses C11 Atomics for synchronisation,
with particular emphasis given to the use of an atomic integer
called \verb"go_ahead" when acquiring thread queue resources.

Here we focus on the semantics of the C11 Atomics features used in this
part of the RTEMS code.

We find them as follows:
\begin{verbatim}
cpukit/include/rtems/score/threadqimpl.h
470:  _Atomic_Store_uint( &gate->go_ahead, 0, ATOMIC_ORDER_RELAXED );
496:  _Atomic_Store_uint( &gate->go_ahead, 1, ATOMIC_ORDER_RELAXED );
510:  while ( _Atomic_Load_uint( &gate->go_ahead, ATOMIC_ORDER_RELAXED ) == 0 ) {

cpukit/include/rtems/score/smplockticket.h
42:  Atomic_Uint next_ticket;
43:  Atomic_Uint now_serving;
66:  _Atomic_Init_uint( &lock->next_ticket, 0U );
67:  _Atomic_Init_uint( &lock->now_serving, 0U );
108:    _Atomic_Fetch_add_uint( &lock->next_ticket, 1U, ATOMIC_ORDER_RELAXED );
112:    _Atomic_Load_uint( &lock->now_serving, ATOMIC_ORDER_ACQUIRE );
120:        _Atomic_Load_uint( &lock->now_serving, ATOMIC_ORDER_ACQUIRE );
169:    _Atomic_Load_uint( &lock->now_serving, ATOMIC_ORDER_RELAXED );
176:  _Atomic_Store_uint( &lock->now_serving, next_ticket, ATOMIC_ORDER_RELEASE );

cpukit/include/rtems/score/threadimpl.h
2185:  _Atomic_Store_uint( &the_thread->Wait.flags, flags, ATOMIC_ORDER_RELAXED );
2203:  return _Atomic_Load_uint( &the_thread->Wait.flags, ATOMIC_ORDER_RELAXED );
2221:  return _Atomic_Load_uint( &the_thread->Wait.flags, ATOMIC_ORDER_ACQUIRE );
2252:  return _Atomic_Compare_exchange_uint(
2290:  return _Atomic_Compare_exchange_uint(

cpukit/include/rtems/score/threadq.h
120:  Atomic_Uint go_ahead;
\end{verbatim}

We initially focus on the use of type \verb"Atomic_Uint",
functions \verb"_Atomic_Store_uint",
and \verb"_Atomic_Load_uint",
and mode \verb"ATOMIC_ORDER_RELAXED",
as used with \verb"go_ahead".

\newpage
\subsection{RTEMS Atomic Usage Mapping}

We notice that the types and calls in the RTEMS sources
are not those in the C11 standard.
They are implemented as follows:

\subsubsection{Atomic uint}

\begin{verbatim}
include/rtems/score/atomic.h
38:typedef CPU_atomic_Uint Atomic_Uint;

include/rtems/score/cpustdatomic.h
56:typedef std::atomic_uint CPU_atomic_Uint; // THIS ONE
86:typedef atomic_uint CPU_atomic_Uint;
116:typedef unsigned int CPU_atomic_Uint;
\end{verbatim}

So, \verb"Atomic_Uint"  is (std::)\verb"atomic_uint".

\subsubsection{ATOMIC ORDER RELAXED}

\begin{verbatim}
include/rtems/score/atomic.h
48:#define ATOMIC_ORDER_RELAXED CPU_ATOMIC_ORDER_RELAXED

include/rtems/score/cpustdatomic.h
66:#define CPU_ATOMIC_ORDER_RELAXED std::memory_order_relaxed // THIS ONE
96:#define CPU_ATOMIC_ORDER_RELAXED memory_order_relaxed
126:#define CPU_ATOMIC_ORDER_RELAXED 0
\end{verbatim}

So, \verb"ATOMIC_ORDER_RELAXED"  is (std::)\verb"memory_order_relaxed".

\subsubsection{Atomic Store uint}

\begin{verbatim}
include/rtems/score/atomic.h
86:#define _Atomic_Store_uint( obj, desr, order ) \
      _CPU_atomic_Store_uint( obj, desr, order )

include/rtems/score/cpustdatomic.h
296:
static inline void _CPU_atomic_Store_uint(
            CPU_atomic_Uint *obj, unsigned int desired, CPU_atomic_Order order )
#if defined(_RTEMS_SCORE_CPUSTDATOMIC_USE_ATOMIC)
  obj->store( desired );
#elif defined(_RTEMS_SCORE_CPUSTDATOMIC_USE_STDATOMIC)
  atomic_store_explicit( obj, desired, order );
#else
  (void) order;
  RTEMS_COMPILER_MEMORY_BARRIER();
  *obj = desired;
#endif
\end{verbatim}

So, \verb"_Atomic_Store_uint"  is probably \verb"atomic_store_explicit".

\subsubsection{Atomic Load uint}

\begin{verbatim}
include/rtems/score/atomic.h
77:
#define _Atomic_Load_uint( obj, order ) \
  _CPU_atomic_Load_uint( obj, order )

include/rtems/score/cpustdatomic.h
222:static inline unsigned int _CPU_atomic_Load_uint(
                            const CPU_atomic_Uint *obj, CPU_atomic_Order order )
{
#if defined(_RTEMS_SCORE_CPUSTDATOMIC_USE_ATOMIC)
  return obj->load( order );
#elif defined(_RTEMS_SCORE_CPUSTDATOMIC_USE_STDATOMIC)
  return atomic_load_explicit( obj, order );
#else
  unsigned int val;

  (void) order;
  val = *obj;
  RTEMS_COMPILER_MEMORY_BARRIER();

  return val;
#endif
}
\end{verbatim}

So, \verb"_Atomic_Load_uint"  is probably \verb"atomic_load_explicit".


\newpage
\subsection{C11 Memory Models}

\newpage
\subsection{C11 Atomics}

Main reference: \url{https://en.cppreference.com/w/c/atomic}.


\subsubsection{Memory Ordering}

From \url{https://en.cppreference.com/w/c/atomic/memory_order#Relaxed_ordering}
we have:
\begin{quote}
  ``
  Atomic operations tagged \verb"memory_order_relaxed"
  are not synchronization operations;
  they do not impose an order among concurrent memory accesses.
  They only guarantee atomicity and modification order consistency.
  ''
\end{quote}
and
\begin{quote}
  ``Typical use for relaxed memory ordering is incrementing counters,
  such as the reference counters,
  since this only requires atomicity,
  but not ordering or synchronization
  (note that decrementing the \verb"shared_ptr" counters
  requires acquire-release synchronization with the destructor)''
\end{quote}

From \url{https://en.cppreference.com/w/c/atomic/memory_order#Release-Acquire_ordering}
we have:
\begin{quote}
  ``If an atomic store in thread A is tagged \verb"memory_order_release"
  and an atomic load in thread B from the same variable
  is tagged \verb"memory_order_acquire",
  all memory writes (non-atomic and relaxed atomic) that happened-before
  the atomic store from the point of view of thread A,
  become visible side-effects in thread B.
  That is, once the atomic load is completed,
  thread B is guaranteed to see everything thread A wrote to memory.''
\end{quote}
and
\begin{quote}
  ``On strongly-ordered systems
  — x86, SPARC TSO, IBM mainframe, etc.
  — release-acquire ordering is automatic for the majority of operations.
  No additional CPU instructions are issued for this synchronization mode;
  only certain compiler optimizations are affected
  (e.g., the compiler is prohibited from moving non-atomic stores
  past the atomic store-release
  or performing non-atomic loads earlier than the atomic load-acquire). ''
\end{quote}

See also
\url{https://en.cppreference.com/w/cpp/language/memory_model#Threads_and_data_races}.


\subsubsection{Atomic Types}

\url{https://en.cppreference.com/w/c/language/atomic}.

Typedef name \verb"atomic_uint" is short for \verb"_Atomic unsigned int".

\subsubsection{Atomic Operations}


\paragraph{Atomic Store Explicit}

\url{https://en.cppreference.com/w/c/atomic/atomic_store}.

\begin{nicec}
void atomic_store_explicit( volatile A* obj, C desired, memory_order order );
\end{nicec}
Atomically replaces the value of the atomic variable pointed to by \texttt{obj}
with \texttt{desired}.
The operation is atomic write operation,
and orders memory accesses according to \texttt{order}.


\paragraph{Atomic Load Explicit}

\url{https://en.cppreference.com/w/c/atomic/atomic_load}.

\begin{nicec}
C atomic_load_explicit( const volatile A* obj, memory_order order );
\end{nicec}
Atomically loads and returns the current value of
the atomic variable pointed to by \texttt{obj}.
The operation is atomic read operation,
and  orders memory accesses according to \texttt{order}.
