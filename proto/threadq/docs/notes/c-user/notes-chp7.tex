\subsection{Chapter 7 Task Manager}

\subsubsection{Introduction}


\subsubsection{Background}

p79:

From RTEMS’ perspective,
a task is the smallest thread of execution
which can compete on its own for system resources.
A task is manifested by the existence of a task control block (TCB).

A TCB is allocated upon creation of the task and is returned to
the TCB free list upon deletion of the task.

The TCB’s elements are modified as a result of system calls
made by the application in response to external and internal stimuli.

The TCB contains a task’s name,
ID, current priority, current and starting states, execution mode, TCB user extension pointer,
scheduling control structures, as well as data required by a blocked task.

A task’s context is stored in the TCB when a task switch occurs.
\dots
When a task is restarted,
the initial state of the task is restored
from the starting context area in the task’s TCB.

The system uses two separate memory areas to manage a task. TCB, and
the other memory area is allocated from the stack space or provided
by the user and contains
\begin{itemize}
  \item
    the task stack,
  \item
    the thread-local storage (TLS), and
  \item
    an optional architecture-specific floating-point context.
\end{itemize}

p80:

A task may exist in one of the following five states:
\begin{itemize}
  \item
    executing - Currently scheduled to the CPU
  \item
    ready - May be scheduled to the CPU
  \item
    blocked - Unable to be scheduled to the CPU
  \item
    dormant - Created task that is not started
  \item
    non-existent - Uncreated or deleted task
\end{itemize}

RTEMS supports 255 levels of priority ranging from 1 to 255.
1 is highest priority.

p81:

A task’s execution mode is a combination of the following four components:
\begin{itemize}
  \item
    preemption
  \item
    ASR processing
  \item
    timeslicing
  \item
    interrupt level
\end{itemize}

The data type \verb"rtems_task_mode" is used to manage the task execution mode.

If preemption is enabled (\verb"RTEMS_PREEMPT")
and a higher priority task is made ready,
then the processor will be taken away from the current task immediately
and given to the higher priority task.

The timeslicing component is used by the RTEMS scheduler to determine how the processor is
allocated to tasks of equal priority.

The asynchronous signal processing component is used
to determine when received signals are to be processed by the task.

The interrupt level component is used to determine
which interrupts will be enabled when the task is executing.

\subsubsection{Operations}

p85:

Newly created tasks are initially placed in the dormant state.
All RTEMS tasks execute in the most privileged mode of the processor

The \verb"rtems_task_start" directive is used
to place a dormant task in the ready state.

A task cannot be restarted unless it has previously been started
(i.e. dormant tasks cannot be restarted).
All restarted tasks are placed in the ready state.

This implies that a task may be suspended as well as blocked waiting
either to acquire a resource or for the expiration of a timer.

p85--86:

The \verb"rtems_task_resume" directive is used to remove another task
from the suspended state.
If the task is not also blocked,
resuming it will place it in the ready state,
allowing it to once again compete for the processor and resources.
If the task was blocked as well as suspended,
this directive clears the suspension and leaves the task in the blocked state.

p86:

The \verb"rtems_task_wake_when" directive creates a sleep timer
which allows a task to go to sleep until a specified date and time.
The calling task is blocked until the specified date and time has occurred,
at which time the task is unblocked.
