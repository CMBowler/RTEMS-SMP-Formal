\subsection{Chapter 26 Multiprocessing Manager}

\textbf{Note:}
\textsf{
At the weekly meeting of 16th March 2021,
Sebastian Huber said that this Manager is to be
considered deprecated,
and that the ``node'' concept no longer really applies,
and has been replaced in the SMP enhancements by the
notion of ``processor''.
}

\subsubsection{Introduction}

p514:

RTEMS allows the entire system,
both hardware and software,
to be viewed logically as a single system.

\subsubsection{Background}

p515:

In keeping with RTEMS philosophy
of providing transparent physical node boundaries,
the minimal heterogeneous processing required is isolated in the MPCI layer.

A processor in a RTEMS system is referred to as a node.
Each node is assigned a unique nonzero node number by the application designer.
RTEMS assumes that node numbers are assigned consecutively
from one to the \verb"maximum_nodes" configuration parameter.

RTEMS assumes that node numbers are assigned
consecutively from one to the \verb"maximum_nodes" configuration parameter.

The Multiprocessor Communications Interface Layer (MPCI)
must be able to route messages based on the node number.

All RTEMS objects which are created with the GLOBAL attribute
will be known on all other nodes.

A task does not have to be global
to perform operations involving remote objects.

The distribution of tasks to processors is performed
during the application design phase.
Dynamic task relocation is not supported by RTEMS.

p516:

RTEMS maintains two tables containing object information
on every node in a multiprocessor system:
a local object table and a global object table.
The local object table on each node is unique
and contains information for all objects created on this node
whether those objects are local or global.
The global object table contains information regarding
all global objects in the system and, consequently, is the same on every node.

To maintain consistency among the table copies,
every node in the system must be informed
of the creation or deletion of a global object.

When an application performs an operation on a remote global object,
RTEMS must generate a Remote Request (RQ) message
and send it to the appropriate node.
After completing the requested operation,
the remote node will build a Remote Response (RR) message
and send it to the originating node.
Messages generated as a side-effect of a directive
(such as deleting a global task)
are known as Remote Processes (RP)
and do not require the receiving node to respond.

p517:

If an uncorrectable error occurs in the user-provided MPCI layer,
the fatal error handler should be invoked.
RTEMS assumes the reliable transmission and reception of messages
by the MPCI and makes no attempt to detect or correct errors.

A proxy is an RTEMS data structure which resides on a remote node
and is used to represent a task which must block as part of a remote operation.
This action can occur as part of the \verb"rtems_semaphore_obtain"
and \verb"rtems_message_queue_receive" directives.
If the object were local,
the task’s control block would be available for modification
to indicate it was blocking on a message queue or semaphore.
However,
the task’s control block resides only on the same node as the task.
As a result,
the remote node must allocate a proxy to represent the task
until it can be readied.

\subsubsection{Multiprocessor Communications Interface Layer}

p518:

The Multiprocessor Communications Interface Layer (MPCI)
is a set of user-provided procedures
which enable the nodes in a multiprocessor system
to communicate with one another.

A packet is sent by RTEMS in each of the following situations:
\begin{itemize}
  \item an RQ is generated on an originating node;
  \item an RR is generated on a destination node;
  \item a global object is created;
  \item a global object is deleted;
  \item a local task blocked on a remote object is deleted;
  \item during system initialization to check for system consistency.
\end{itemize}

\subsubsection{Operations}


p522:

The \verb"rtems_multiprocessing_announce" directive is called by the MPCI layer
to inform RTEMS that a packet has arrived from another node.
This directive can be called from
an interrupt service routine
or from within a polling routine.
