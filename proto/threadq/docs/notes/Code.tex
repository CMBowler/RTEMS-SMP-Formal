\section{Code Notes}

Wanted: call sequence paths from
\verb"rtems_semaphore_obtain()"
and
\verb"rtems_semaphore_release()",
via
\verb"_Thread_queue_Enqueue()"
and
\verb"_Thread_queue_Surrender()"
to
\verb"_Thread_Wait_acquire_critical".

The routing may also involve:
\verb"_Thread_queue_Gate_add",
\verb"_Thread_queue_Gate_wait",
and
\verb"_Thread_queue_Gate_open".

\subsection{Creating a Semaphore}

This is setup by a call to \verb"rtems_semaphore_create"
with options for
\begin{description}
  \item [queue discipline] \verb"RTEMS_PRIORITY"
  \item [semaphore class] \verb"RTEMS_BINARY_SEMAPHORE"
  \item [locking protocol] \verb"RTEMS_INHERIT_PRIORITY"
\end{description}
There is a \verb"SEMAPHORE_KIND_MASK"
that contains \verb"RTEMS_INHERIT_PRIORITY".
\begin{nicec}
attr = RTEMS_BINARY_SEMAPHORE |  RTEMS_PRIORITY | RTEMS_INHERIT_PRIORITY ;
rc = rtems_semaphore_create( "SEMA", 1, attr, _ &semId );
\end{nicec}
Internally:
\begin{nicec}
#define RTEMS_SEMAPHORE_CLASS 0x00000030 // (incl. RTEMS_BINARY_SEMAPHORE )
#define SEMAPHORE_KIND_MASK ( RTEMS_SEMAPHORE_CLASS | RTEMS_INHERIT_PRIORITY \
  | RTEMS_PRIORITY_CEILING | RTEMS_MULTIPROCESSOR_RESOURCE_SHARING )

  maybe_global = RTEMS_INHERIT_PRIORITY; // attribute_set & SEMAPHORE_KIND_MASK
  mutex_with_protocol =  attr ;
    // attribute_set & ( SEMAPHORE_KIND_MASK | RTEMS_GLOBAL | RTEMS_PRIORITY )

  } else if (
    mutex_with_protocol
      == ( RTEMS_BINARY_SEMAPHORE | RTEMS_PRIORITY | RTEMS_INHERIT_PRIORITY )
  ) {
    variant = SEMAPHORE_VARIANT_MUTEX_INHERIT_PRIORITY;  // ... }

  the_semaphore = _Semaphore_Allocate();
  // include/rtems/rtems/semimpl.h
  flags = SEMAPHORE_VARIANT_MUTEX_INHERIT_PRIORITY;
         // _Semaphore_Set_variant( 0, variant );
  if ( _Attributes_Is_priority( attribute_set ) ) {
    flags = _Semaphore_Set_discipline( flags, SEMAPHORE_DISCIPLINE_PRIORITY );
  _Semaphore_Set_flags( the_semaphore, flags );
  executing = _Thread_Get_executing();
  switch ( variant ) {
    case SEMAPHORE_VARIANT_MUTEX_INHERIT_PRIORITY:
      _CORE_recursive_mutex_Initialize(
        &the_semaphore->Core_control.Mutex.Recursive
      );
      if ( count == 0 ) {
        _CORE_mutex_Set_owner(
          &the_semaphore->Core_control.Mutex.Recursive.Mutex,
          executing
        );
        _Thread_Resource_count_increment( executing );
      }
      status = STATUS_SUCCESSFUL;
      break;
  }
  _Objects_Open(
    &_Semaphore_Information,
    &the_semaphore->Object,
    (Objects_Name) name
  );
  *id = the_semaphore->Object.id;
  _Objects_Allocator_unlock();
  return RTEMS_SUCCESSFUL;
}
\end{nicec}

So
 variable \verb"maybe_global" will contain \verb"RTEMS_INHERIT_PRIORITY",
 and \verb"mutex_with_protocol" will contain \verb"RTEMS_PRIORITY".

 Given these, then we set \verb"variant"
 to \verb"SEMAPHORE_VARIANT_MUTEX_INHERIT_PRIORITY".


From \texttt{coremuteximpl.h}:
\begin{nicec}
// 49:
RTEMS_INLINE_ROUTINE void _CORE_mutex_Initialize(
  CORE_mutex_Control *the_mutex
)
{
  _Thread_queue_Object_initialize( &the_mutex->Wait_queue );
}

// 182:
RTEMS_INLINE_ROUTINE void _CORE_recursive_mutex_Initialize(
  CORE_recursive_mutex_Control *the_mutex
)
{
  _CORE_mutex_Initialize( &the_mutex->Mutex );
  the_mutex->nest_level = 0;
}
\end{nicec}

From: \texttt{threadq.c}:
\begin{nicec}
// 144:
void _Thread_queue_Initialize(
  Thread_queue_Control *the_thread_queue,
  const char           *name
)
{
  _Thread_queue_Queue_initialize( &the_thread_queue->Queue, name );
#if defined(RTEMS_SMP)
  _SMP_lock_Stats_initialize( &the_thread_queue->Lock_stats, "Thread Queue" );
#endif
}
// 155:
void _Thread_queue_Object_initialize( Thread_queue_Control *the_thread_queue )
{
  _Thread_queue_Initialize( the_thread_queue, _Thread_queue_Object_name );
}
\end{nicec}

From \texttt{threadqimpl.h}:
\begin{nicec}
// 545:
RTEMS_INLINE_ROUTINE void _Thread_queue_Queue_initialize(
  Thread_queue_Queue *queue,
  const char         *name
)
{
#if defined(RTEMS_SMP)
  _SMP_ticket_lock_Initialize( &queue->Lock );
#endif
  queue->heads = NULL;
  queue->owner = NULL;
  queue->name = name;
}
\end{nicec}

\subsection{Obtaining a Semaphore}

We assume a binary, priority inherit semaphore,
as created above.

We consider a call to obtain a semaphore that will wait (forever)
\begin{nicec}
rc = rtems_semaphore_obtain( semId, RTEMS_WAIT, RTEMS_NO_TIMEOUT );
\end{nicec}
Internally:
\begin{nicec}
  the_semaphore = _Semaphore_Get( semId, &queue_context );
  executing = _Thread_Executing;
  // cpukit/include/rtems/rtems/optionsimpl.h
  wait = true; // !_Options_Is_no_wait( RTEMS_WAIT );
  if ( wait ) {
    // _Thread_queue_Context_set_enqueue_timeout_ticks( &queue_context, timeout );
    queue_context->Timeout.ticks = ticks;
    queue_context->enqueue_callout = _Thread_queue_Add_timeout_ticks;

  flags = _Semaphore_Get_flags( the_semaphore );
  variant = SEMAPHORE_VARIANT_MUTEX_INHERIT_PRIORITY
            // _Semaphore_Get_variant( flags );

  switch ( variant ) {
    case SEMAPHORE_VARIANT_MUTEX_INHERIT_PRIORITY:
      status = _CORE_recursive_mutex_Seize(
        &the_semaphore->Core_control.Mutex.Recursive,
        CORE_MUTEX_TQ_PRIORITY_INHERIT_OPERATIONS,
        executing,
        wait,
        _CORE_recursive_mutex_Seize_nested,
        &queue_context
      );
      break;
  }
  return _Status_Get( status );
\end{nicec}

Now, seizing that mutex:
\begin{nicec}
/**
 * @brief Seizes the recursive mutex.
 *
 * @param[in, out] the_mutex The recursive mutex to seize.
 * @param operations The thread queue operations.
 * @param[out] executing The executing thread.
 * @param wait Indicates whether the calling thread is willing to wait.
 * @param nested Returns the status of a recursive mutex.
 * @param queue_context The thread queue context.
 *
 * @retval STATUS_SUCCESSFUL The owner of the mutex was NULL, successful
 *      seizing of the mutex.
 * @retval _Thread_Wait_get_status The status of the executing thread.
 * @retval STATUS_UNAVAILABLE The calling thread is not willing to wait.
 */
RTEMS_INLINE_ROUTINE Status_Control _CORE_recursive_mutex_Seize(
  CORE_recursive_mutex_Control  *the_mutex,
  const Thread_queue_Operations *operations,
  Thread_Control                *executing,
  bool                           wait,
  Status_Control              ( *nested )( CORE_recursive_mutex_Control * ),
  Thread_queue_Context          *queue_context
)
{
  Thread_Control *owner;

  _CORE_mutex_Acquire_critical( &the_mutex->Mutex, queue_context );

  owner = _CORE_mutex_Get_owner( &the_mutex->Mutex );

  if ( owner == NULL ) {
    _CORE_mutex_Set_owner( &the_mutex->Mutex, executing );
    _Thread_Resource_count_increment( executing );
    _CORE_mutex_Release( &the_mutex->Mutex, queue_context );
    return STATUS_SUCCESSFUL;
  }

  if ( owner == executing ) {
    Status_Control status;

    status = ( *nested )( the_mutex );
    _CORE_mutex_Release( &the_mutex->Mutex, queue_context );
    return status;
  }

  return _CORE_mutex_Seize_slow(
    &the_mutex->Mutex,
    operations,
    executing,
    wait,
    queue_context
  );
}
\end{nicec}


\coremutexseizeC:
\begin{nicec}
Status_Control _CORE_mutex_Seize_slow(
  CORE_mutex_Control            *the_mutex,
  const Thread_queue_Operations *operations,
  Thread_Control                *executing,
  bool                           wait,
  Thread_queue_Context          *queue_context
)
{
  if ( wait ) {
    _Thread_queue_Context_set_thread_state(
      queue_context,
      STATES_WAITING_FOR_MUTEX
    );
    _Thread_queue_Context_set_deadlock_callout(
      queue_context,
      _Thread_queue_Deadlock_status
    );
    _Thread_queue_Enqueue(
      &the_mutex->Wait_queue.Queue,
      operations,
      executing,
      queue_context
    );
    return _Thread_Wait_get_status( executing );
  } else {
    _CORE_mutex_Release( the_mutex, queue_context );
    return STATUS_UNAVAILABLE;
  }
}
\end{nicec}

\threadqenqueueC:
\begin{nicec}
void _Thread_queue_Enqueue(
  Thread_queue_Queue            *queue,
  const Thread_queue_Operations *operations,
  Thread_Control                *the_thread,
  Thread_queue_Context          *queue_context
)
{
  Per_CPU_Control *cpu_self;
  bool             success;

  _Assert( queue_context->enqueue_callout != NULL );

  _Thread_Wait_claim( the_thread, queue );

  if ( !_Thread_queue_Path_acquire_critical( queue, the_thread, queue_context ) ) {
    _Thread_queue_Path_release_critical( queue_context );
    _Thread_Wait_restore_default( the_thread );
    _Thread_queue_Queue_release( queue, &queue_context->Lock_context.Lock_context );
    _Thread_Wait_tranquilize( the_thread );
    _Assert( queue_context->deadlock_callout != NULL );
    ( *queue_context->deadlock_callout )( the_thread );
    return;
  }

  _Thread_queue_Context_clear_priority_updates( queue_context );
  _Thread_Wait_claim_finalize( the_thread, operations );
  ( *operations->enqueue )( queue, the_thread, queue_context );

  _Thread_queue_Path_release_critical( queue_context );

  the_thread->Wait.return_code = STATUS_SUCCESSFUL;
  _Thread_Wait_flags_set( the_thread, THREAD_QUEUE_INTEND_TO_BLOCK );
  cpu_self = _Thread_queue_Dispatch_disable( queue_context );
  _Thread_queue_Queue_release( queue, &queue_context->Lock_context.Lock_context );

  ( *queue_context->enqueue_callout )(
    queue,
    the_thread,
    cpu_self,
    queue_context
  );

  /*
   *  Set the blocking state for this thread queue in the thread.
   */
  _Thread_Set_state( the_thread, queue_context->thread_state );

  /*
   * At this point thread dispatching is disabled, however, we already released
   * the thread queue lock.  Thus, interrupts or threads on other processors
   * may already changed our state with respect to the thread queue object.
   * The request could be satisfied or timed out.  This situation is indicated
   * by the thread wait flags.  Other parties must not modify our thread state
   * as long as we are in the THREAD_QUEUE_INTEND_TO_BLOCK thread wait state,
   * thus we have to cancel the blocking operation ourself if necessary.
   */
  success = _Thread_Wait_flags_try_change_acquire(
    the_thread,
    THREAD_QUEUE_INTEND_TO_BLOCK,
    THREAD_QUEUE_BLOCKED
  );
  if ( !success ) {
    _Thread_Remove_timer_and_unblock( the_thread, queue );
  }

  _Thread_Priority_update( queue_context );
  _Thread_Dispatch_direct( cpu_self );
}
\end{nicec}
The loop whose termination  is an issue can be found in
\verb"_Thread_queue_Path_acquire_critical",
also in the same file.

\begin{nicec}
#if !defined(RTEMS_SMP)
static
#endif
bool _Thread_queue_Path_acquire_critical(
  Thread_queue_Queue   *queue,
  Thread_Control       *the_thread,
  Thread_queue_Context *queue_context
)
{
  Thread_Control     *owner;
#if defined(RTEMS_SMP)
  Thread_queue_Link  *link;
  Thread_queue_Queue *target;

  /*
   * For an overview please look at the non-SMP part below.  We basically do
   * the same on SMP configurations.  The fact that we may have more than one
   * executing thread and each thread queue has its own SMP lock makes the task
   * a bit more difficult.  We have to avoid deadlocks at SMP lock level, since
   * this would result in an unrecoverable deadlock of the overall system.
   */

  _Chain_Initialize_empty( &queue_context->Path.Links );

  owner = queue->owner;

  if ( owner == NULL ) {
    return true;
  }

  if ( owner == the_thread ) {
    return false;
  }

  _Chain_Initialize_node(
    &queue_context->Path.Start.Lock_context.Wait.Gate.Node
  );
  link = &queue_context->Path.Start;
  _RBTree_Initialize_node( &link->Registry_node );
  _Chain_Initialize_node( &link->Path_node );

  do {
    _Chain_Append_unprotected( &queue_context->Path.Links, &link->Path_node );
    link->owner = owner;

    _Thread_Wait_acquire_default_critical(
      owner,
      &link->Lock_context.Lock_context
    );

    target = owner->Wait.queue;
    link->Lock_context.Wait.queue = target;

    if ( target != NULL ) {
      if ( _Thread_queue_Link_add( link, queue, target ) ) {
        _Thread_queue_Gate_add(
          &owner->Wait.Lock.Pending_requests,
          &link->Lock_context.Wait.Gate
        );
        _Thread_Wait_release_default_critical(
          owner,
          &link->Lock_context.Lock_context
        );
        _Thread_Wait_acquire_queue_critical( target, &link->Lock_context );

        if ( link->Lock_context.Wait.queue == NULL ) {
          _Thread_queue_Link_remove( link );
          _Thread_Wait_release_queue_critical( target, &link->Lock_context );
          _Thread_Wait_acquire_default_critical(
            owner,
            &link->Lock_context.Lock_context
          );
          _Thread_Wait_remove_request_locked( owner, &link->Lock_context );
          _Assert( owner->Wait.queue == NULL );
          return true;
        }
      } else {
        link->Lock_context.Wait.queue = NULL;
        _Thread_queue_Path_append_deadlock_thread( owner, queue_context );
        return false;
      }
    } else {
      return true;
    }

    link = &owner->Wait.Link;
    queue = target;
    owner = queue->owner;
  } while ( owner != NULL );
#else
  do {
    owner = queue->owner;

    if ( owner == NULL ) {
      return true;
    }

    if ( owner == the_thread ) {
      return false;
    }

    queue = owner->Wait.queue;
  } while ( queue != NULL );
#endif

  return true;
}  
\end{nicec}


\subsection{Lower down\dots}

The code mentioned by Sebastian, from \threadimplH:
\begin{nicec}
/**
 * @brief Acquires the thread wait lock inside a critical section (interrupts
 * disabled).
 *
 * @param[in, out] the_thread The thread.
 * @param[in, out] queue_context The thread queue context for the corresponding
 *   _Thread_Wait_release_critical().
 */
RTEMS_INLINE_ROUTINE void _Thread_Wait_acquire_critical(
  Thread_Control       *the_thread,
  Thread_queue_Context *queue_context
)
{
#if defined(RTEMS_SMP)
  Thread_queue_Queue *queue;

  _Thread_Wait_acquire_default_critical(
    the_thread,
    &queue_context->Lock_context.Lock_context
  );

  queue = the_thread->Wait.queue;
  queue_context->Lock_context.Wait.queue = queue;

  if ( queue != NULL ) {
    _Thread_queue_Gate_add(
      &the_thread->Wait.Lock.Pending_requests,
      &queue_context->Lock_context.Wait.Gate
    );
    _Thread_Wait_release_default_critical(
      the_thread,
      &queue_context->Lock_context.Lock_context
    );
    _Thread_Wait_acquire_queue_critical( queue, &queue_context->Lock_context );

    if ( queue_context->Lock_context.Wait.queue == NULL ) {
      _Thread_Wait_release_queue_critical(
        queue,
        &queue_context->Lock_context
      );
      _Thread_Wait_acquire_default_critical(
        the_thread,
        &queue_context->Lock_context.Lock_context
      );
      _Thread_Wait_remove_request_locked(
        the_thread,
        &queue_context->Lock_context
      );
      _Assert( the_thread->Wait.queue == NULL );
    }
  }
#else
  (void) the_thread;
  (void) queue_context;
#endif
}

\end{nicec}

\newpage\subsection{Code Index}

All paths are relative to \texttt{rtems/cpukit}.


\def\ditto{~~~~~''}

\begin{tabular}{lr@{ : }l}
   \verb"go_ahead" &  120 & \threadqH
\\ \ditto          &  470 & \threadqimplH
\\ \ditto          &  496 & \threadqimplH
\\ \ditto          &  510 & \threadqimplH
\\ \verb"rtems_semaphore_create"            &  218 & \semH
\\ \ditto                                   &   35 & \semcreateH
\\ \verb"rtems_semaphore_obtain"            &   53 & \semobtainC
\\ \verb"rtems_semaphore_release"           &   26 & \semreleaseC
\\ \verb"_Thread_queue_Enqueue"             &  387 & \threadqenqueueC
\\ \verb"_Thread_queue_Surrender"           &  667 & \threadqenqueueC
\\ \verb"_Thread_Wait_acquire_critical"     & 1795 & \threadimplH
\\ \verb"_Thread_Add_timeout_ticks"         & 2388 & \threadimplH
\\ \verb"_Thread_queue_Gate_add"            &  479 & \threadqimplH
\\ \verb"_Thread_queue_Gate_wait"           &  506 & \threadqimplH
\\ \verb"_Thread_queue_Gate_open"           &  492 & \threadqimplH
\\ \verb"_Thread_queue_Do_acquire_critical" &  663 & \threadqimplH
\\ \verb"_Thread_queue_Acquire_critical"    &  679 & \threadqimplH
\\ \verb"_Thread_Resource_count_increment"  & 1322 & \threadimplH
\\ \verb"Thread_queue_Gate"                 &  121 & \threadqH
\\ \verb"THREAD_QUEUE_OBJECT_ASSERT"        & 1452 & \threadqimplH
\\ \verb"RTEMS_STATIC_ASSERT"               &  759 & \basedefsH
\\ \verb"_CORE_recursive_mutex_Initialize"  &  182 & \coremuteximplH
\\ \verb"_CORE_recursive_mutex_Seize"       &  220 & \coremuteximplH
\\ \verb"_CORE_mutex_Acquire_critical"      &   72 & \coremuteximplH
\\ \verb"_CORE_mutex_Get_owner"             &  101 & \coremuteximplH
\\ \verb"_CORE_mutex_Set_owner"             &  152 & \coremuteximplH
\\ \verb"_Thread_queue_Add_timeout_ticks"   &   28 & \threadqtimeoutC
\\ \verb"_CORE_mutex_Release"               &   86 & \coremutexseizeC
\\ \verb"_CORE_mutex_Seize_slow"            &  138 & \coremuteximplH
\\ \ditto                                   &   28 & \coremutexseizeC
\\ \verb"_Thread_queue_Context_set_thread_state" &  176 & \threadqimplH
\\ \verb" _Thread_queue_Context_set_deadlock_callout" &  322 & \threadqimplH
\\ \verb"_Thread_queue_Enqueue" &  387 & \threadqenqueueC
\\ \verb"_Thread_Wait_get_status" & 2337 & \threadimplH
\end{tabular}

