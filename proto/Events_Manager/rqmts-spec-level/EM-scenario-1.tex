\chapter{Event Manager Scenario 1}

We have an initial scenario, generated from a correct model,
with a \texttt{assert(false)} statement added at the end.

The current logged output is in file \texttt{evtmrg001.log}.

We can summarise the scenario as follows:
\begin{enumerate}
  \item
    We have task 0, low priority, pre-emptable,
    doing two sends to task 1, high priority.
    In addition we have clock and ``system'' tasks
    that implement various concurrent behaviours.
  \item
    Task 1 calls \texttt{receive_event} looking for all of $\{4,8\}$ (\texttt{1100})
    and blocks, waiting on these events.
  \item
    Task 0 calls \texttt{send_event} with event set $\{2,8\}$ (\texttt{1010}),
    which does not satisfy task 1, so the call returns
    immediately signalling success.
  \item
    Task 0 calls \texttt{send_event} with event set $\{4\}$ (\texttt{0100}),
    which now results in task 2 being satisfied.
    Task 1 is unblocked,
    and returns Success, with output events set to $\{4,8\}$,
    and its pending events now being $\{2\}$.

    Task 0, being lower priority is pre-empted.
  \item
    The ``system'' task wakes up  Task 0,
    abstracting future system behaviour that would lead to this outcome.
\end{enumerate}

In terms of the pre- and post-conditions, and transitions,
of the Event Manager requirements in \texttt{send-receive.yml},
we note the following:
\begin{itemize}
  \item
    The conditions are defined in a context of both a single
    \texttt{send_event} and \texttt{receive_event}.
    We have to manage two sends.
  \item
    In the above scenario, matters are simplified because the receiver runs
    first and so is blocked for both sends.
  \item
    Pre-condition \textbf{Id} has value \texttt{Task} throughout.
  \item
    Pre-condition \textbf{Sent} has value \texttt{MixedAny} for the first call,
    and \texttt{Any} for the second call.
  \item
    Pre-condition \textbf{ReceiverState} has value \texttt{NotWaiting},
    that becomes(?) \texttt{Intend} as it prepares to run.
    Once it runs, its value becomes \texttt{Higher}
    for the rest of the scenario.
  \item
    Pre-condition \textbf{Satisfy} has value \texttt{All} throughout.
  \item
    Post-condition \textbf{SendStatus} has value \texttt{Ok} throughout.
  \item
    Post-condition \textbf{ReceiveStatus} has value \texttt{Blocked}
    after it is called, and after the first send.
    It changes to \texttt{Satisfied} after the second send.
  \item
    Post-condition \textbf{SenderPreemption} has value \texttt{No}
    after the first send, and becomes \texttt{Yes} after the second.
\end{itemize}

We can now determine that the following transitions apply:
