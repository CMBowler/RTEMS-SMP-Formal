\section{Specification \texttt{send-receive}}

This was generated by \texttt{spec2tex.py}
with input from \texttt{send-receive.yml}.

This version is intended to be edited by hand.


\subsection{Pre-Conditions}

\begin{description}
  \item[Id]~
  \begin{description}
  \item[InvId]
    The id parameter of the send directive shall be an invalid task object
identifier.
\begin{verbatim}
ctx->receiver_id = 0xffffffff;
ctx->sender_type = SENDER_SELF;
\end{verbatim}
  \item[Task]
    The id parameter of the send directive shall be a valid task object
identifier.
\begin{verbatim}
ctx->receiver_id = ctx->runner_id;
\end{verbatim}
  \end{description}

  \item[Send]~
  \begin{description}
  \item[Zero]
    The event set sent shall be the empty.
\begin{verbatim}
ctx->events_to_send = 0;
\end{verbatim}
  \item[Unrelated]
    The event set sent shall be unrelated to the event receive condition.
\begin{verbatim}
ctx->events_to_send = RTEMS_EVENT_7;
\end{verbatim}
  \item[Any]
    The event set sent shall be contain at least one but not all events of
the event receive condition.
\begin{verbatim}
ctx->events_to_send = RTEMS_EVENT_5;
\end{verbatim}
  \item[All]
    The event set sent shall be contain all events of the event receive condition.
\begin{verbatim}
ctx->events_to_send = RTEMS_EVENT_5 | RTEMS_EVENT_23;
\end{verbatim}
  \item[MixedAny]
    The event set sent shall be contain at least one but not all events of
the event receive condition and at least one unrelated event.
\begin{verbatim}
ctx->events_to_send = RTEMS_EVENT_5 | RTEMS_EVENT_7;
\end{verbatim}
  \item[MixedAll]
    The event set sent shall be contain all events of the event receive
condition and at least one unrelated event.
\begin{verbatim}
ctx->events_to_send = RTEMS_EVENT_5 | RTEMS_EVENT_7 | RTEMS_EVENT_23;
\end{verbatim}
  \end{description}

  \item[ReceiverState]~
  \begin{description}
  \item[NotWaiting]
    The receiver task shall not be waiting for events.
\begin{verbatim}
ctx->sender_type = SENDER_SELF;
ctx->receive_type = RECEIVE_SKIP;
\end{verbatim}
  \item[Poll]
    The receiver task shall poll for events.
\begin{verbatim}
ctx->sender_type = SENDER_SELF;
ctx->receive_type = RECEIVE_NORMAL;
ctx->receive_option_set |= RTEMS_NO_WAIT;
\end{verbatim}
  \item[Timeout]
    The receiver task shall have waited for events with a timeout which
occurred.
\begin{verbatim}
ctx->sender_type = SENDER_SELF_2;
ctx->receive_type = RECEIVE_NORMAL;
ctx->receive_timeout = 1;
\end{verbatim}
  \item[Lower]
    The receiver task shall be blocked waiting for events and the receiver
task shall have a lower priority than the sender task.
\begin{verbatim}
ctx->sender_type = SENDER_WORKER;
ctx->sender_prio = PRIO_HIGH;
ctx->receive_type = RECEIVE_NORMAL;
\end{verbatim}
  \item[Equal]
    The receiver task shall be blocked waiting for events and the receiver
task shall have a priority equal to the sender task.
\begin{verbatim}
ctx->sender_type = SENDER_WORKER;
ctx->sender_prio = PRIO_NORMAL;
ctx->receive_type = RECEIVE_NORMAL;
\end{verbatim}
  \item[Higher]
    The receiver task shall be blocked waiting for events and the receiver
task shall have a higher priority than the sender task.
\begin{verbatim}
ctx->sender_type = SENDER_WORKER;
ctx->sender_prio = PRIO_LOW;
ctx->receive_type = RECEIVE_NORMAL;
\end{verbatim}
  \item[Other]
    The receiver task shall be blocked waiting for events and the receiver
task shall be on another scheduler instance than the sender task.
\begin{verbatim}
ctx->sender_type = SENDER_WORKER;
ctx->sender_prio = PRIO_OTHER;
ctx->receive_type = RECEIVE_NORMAL;
\end{verbatim}
  \item[Intend]
    The receiver task shall intend to block for waiting for events.
\begin{verbatim}
ctx->sender_type = SENDER_INTERRUPT;
ctx->receive_type = RECEIVE_INTERRUPT;
\end{verbatim}
  \end{description}

  \item[Satisfy]~
  \begin{description}
  \item[All]
    The receiver task shall be interested in all input events.
\begin{verbatim}
ctx->receive_option_set |= RTEMS_EVENT_ALL;
\end{verbatim}
  \item[Any]
    The receiver task shall be interested in any input event.
\begin{verbatim}
ctx->receive_option_set |= RTEMS_EVENT_ANY;
\end{verbatim}
  \end{description}

\end{description}


\subsection{Post-Conditions}

\begin{description}
  \item[SendStatus]~
  \begin{description}
  \item[Ok]
    The send event status shall be RTEMS\_SUCCESSFUL.
\begin{verbatim}
T_rsc_success( ctx->send_status );
\end{verbatim}
  \item[InvId]
    The send event status shall be RTEMS\_INVALID\_ID.
\begin{verbatim}
T_rsc( ctx->send_status, RTEMS_INVALID_ID );
\end{verbatim}
  \end{description}

  \item[ReceiveStatus]~
  \begin{description}
  \item[None]
    There shall be no pending events.
\begin{verbatim}
T_eq_int( ctx->receive_condition_state, RECEIVE_COND_UNKNOWN );
T_eq_u32( GetPendingEvents( ctx ), 0 );
\end{verbatim}
  \item[Pending]
    All events sent shall be pending.
\begin{verbatim}
T_eq_int( ctx->receive_condition_state, RECEIVE_COND_UNKNOWN );
T_eq_u32( GetPendingEvents( ctx ), ctx->events_to_send );
\end{verbatim}
  \item[Timeout]
    The receive event status shall be RTEMS\_TIMEOUT.  All events sent after
the timeout shall be pending.
\begin{verbatim}
T_rsc( ctx->receive_status, RTEMS_TIMEOUT );
T_eq_int( ctx->receive_condition_state, RECEIVE_COND_UNKNOWN );
T_eq_u32( GetPendingEvents( ctx ), ctx->events_to_send );
\end{verbatim}
  \item[Satisfied]
    The receive event status shall be RTEMS\_SUCCESSFUL.  The received events
shall be equal to the input events sent.  The pending events shall be
equal to the events sent which are not included in the input events.
\begin{verbatim}
T_rsc( ctx->receive_status, RTEMS_SUCCESSFUL );

if ( ctx->receive_type != RECEIVE_NORMAL ) {
  T_eq_int( ctx->receive_condition_state, RECEIVE_COND_SATSIFIED );
}

T_eq_u32( ctx->received_events, ctx->events_to_send & INPUT_EVENTS );
T_eq_u32( GetPendingEvents( ctx ), ctx->events_to_send & ~INPUT_EVENTS );
\end{verbatim}
  \item[Unsatisfied]
    The receive event status shall be RTEMS\_UNSATISFIED.  All sent events
shall be pending.
\begin{verbatim}
T_rsc( ctx->receive_status, RTEMS_UNSATISFIED );
T_eq_int( ctx->receive_condition_state, RECEIVE_COND_UNKNOWN );
T_eq_u32( GetPendingEvents( ctx ), ctx->events_to_send );
\end{verbatim}
  \item[Blocked]
    The receiver task shall remain blocked waiting for events after the
directive call.  All sent events shall be pending.
\begin{verbatim}
T_eq_int( ctx->receive_condition_state, RECEIVE_COND_UNSATISFIED );
T_eq_u32( ctx->unsatisfied_pending, ctx->events_to_send );
\end{verbatim}
  \end{description}

  \item[SenderPreemption]~
  \begin{description}
  \item[No]
    There shall be no sender preemption.
\begin{verbatim}
/*
 * There may be a thread switch to the runner thread if the sender thread
 * was on another scheduler instance.
 */

T_le_sz( log->recorded, 1 );

for ( i = 0; i < log->recorded; ++i ) {
  T_ne_u32( log->events[ i ].executing, ctx->worker_id );
  T_eq_u32( log->events[ i ].heir, ctx->runner_id );
}
\end{verbatim}
  \item[Yes]
    There shall be a sender preemption.
\begin{verbatim}
T_eq_sz( log->recorded, 2 );
T_eq_u32( log->events[ 0 ].heir, ctx->runner_id );
T_eq_u32( log->events[ 1 ].heir, ctx->worker_id );
\end{verbatim}
  \end{description}

\end{description}


\subsection{Skip Reasons}

\begin{description}
  \item[NoOtherScheduler]~
    In non-SMP configurations, there exists exactly one scheduler instance.
\end{description}

\newpage


\subsection{Transition Maps}



\subsubsection{Transition 1}


\textsc{Before:}
\newline.~~~~$Id=InvId$
\newline\textsc{After:}
\newline.~~~~ReceiveStatus=None
\newline.~~~~SendStatus=InvId
\newline.~~~~SenderPreemption=No

\subsubsection{Transition 2}


\textsc{Before:}
\newline.~~~~$Id=Task$
\newline.~~~~$ReceiverState=NotWaiting$
\newline.~~~~$Send=all$
\newline\textsc{After:}
\newline.~~~~ReceiveStatus=Pending
\newline.~~~~SendStatus=Ok
\newline.~~~~SenderPreemption=No

\subsubsection{Transition 3}


\textsc{Before:}
\newline.~~~~$Id=Task$
\newline.~~~~$ReceiverState=Timeout$
\newline.~~~~$Satisfy=all$
\newline.~~~~$Send=all$
\newline\textsc{After:}
\newline.~~~~ReceiveStatus=Timeout
\newline.~~~~SendStatus=Ok
\newline.~~~~SenderPreemption=No

\subsubsection{Transition 4}


\textsc{Before:}
\newline.~~~~$Id=Task$
\newline.~~~~$ReceiverState=Poll$
\newline.~~~~$Satisfy=all$
\newline.~~~~$Send \in \{Zero,Unrelated\}$
\newline\textsc{After:}
\newline.~~~~ReceiveStatus=Unsatisfied
\newline.~~~~SendStatus=Ok
\newline.~~~~SenderPreemption=No

\subsubsection{Transition 5}


\textsc{Before:}
\newline.~~~~$Id=Task$
\newline.~~~~$ReceiverState \in \{Lower,Equal,Higher,Intend\}$
\newline.~~~~$Satisfy=all$
\newline.~~~~$Send \in \{Unrelated,Zero\}$
\newline\textsc{After:}
\newline.~~~~ReceiveStatus=Blocked
\newline.~~~~SendStatus=Ok
\newline.~~~~SenderPreemption=No

\subsubsection{Transition 6}


\textsc{Before:}
\newline.~~~~$Id=Task$
\newline.~~~~$ReceiverState=Higher$
\newline.~~~~$Satisfy=all$
\newline.~~~~$Send \in \{All,MixedAll\}$
\newline\textsc{After:}
\newline.~~~~ReceiveStatus=Satisfied
\newline.~~~~SendStatus=Ok
\newline.~~~~SenderPreemption=Yes

\subsubsection{Transition 7}


\textsc{Before:}
\newline.~~~~$Id=Task$
\newline.~~~~$ReceiverState \in \{Poll,Lower,Equal,Intend\}$
\newline.~~~~$Satisfy=all$
\newline.~~~~$Send \in \{All,MixedAll\}$
\newline\textsc{After:}
\newline.~~~~ReceiveStatus=Satisfied
\newline.~~~~SendStatus=Ok
\newline.~~~~SenderPreemption=No

\subsubsection{Transition 8}


\textsc{Before:}
\newline.~~~~$Id=Task$
\newline.~~~~$ReceiverState=Higher$
\newline.~~~~$Satisfy=Any$
\newline.~~~~$Send \in \{Any,MixedAny\}$
\newline\textsc{After:}
\newline.~~~~ReceiveStatus=Satisfied
\newline.~~~~SendStatus=Ok
\newline.~~~~SenderPreemption=Yes

\subsubsection{Transition 9}


\textsc{Before:}
\newline.~~~~$Id=Task$
\newline.~~~~$ReceiverState \in \{Poll,Lower,Equal,Intend\}$
\newline.~~~~$Satisfy=Any$
\newline.~~~~$Send \in \{Any,MixedAny\}$
\newline\textsc{After:}
\newline.~~~~ReceiveStatus=Satisfied
\newline.~~~~SendStatus=Ok
\newline.~~~~SenderPreemption=No

\subsubsection{Transition 10}


\textsc{Before:}
\newline.~~~~$Id=Task$
\newline.~~~~$ReceiverState=Poll$
\newline.~~~~$Satisfy=All$
\newline.~~~~$Send \in \{Any,MixedAny\}$
\newline\textsc{After:}
\newline.~~~~ReceiveStatus=Unsatisfied
\newline.~~~~SendStatus=Ok
\newline.~~~~SenderPreemption=No

\subsubsection{Transition 11}


\textsc{Before:}
\newline.~~~~$Id=Task$
\newline.~~~~$ReceiverState \in \{Lower,Equal,Higher,Intend\}$
\newline.~~~~$Satisfy=All$
\newline.~~~~$Send \in \{Any,MixedAny\}$
\newline\textsc{After:}
\newline.~~~~ReceiveStatus=Blocked
\newline.~~~~SendStatus=Ok
\newline.~~~~SenderPreemption=No

\subsubsection{Transition 12}


\textsc{Before:}
\newline.~~~~$Id=Task$
\newline.~~~~$ReceiverState=Other$
\newline.~~~~$Satisfy=all$
\newline.~~~~$Send=all$
\newline\textsc{After:}
\newline~NoOtherScheduler

\subsubsection{Transition 13}


\textsc{Before:}
\newline.~~~~$Id=Task$
\newline.~~~~$ReceiverState=Other$
\newline.~~~~$Satisfy=all$
\newline.~~~~$Send \in \{Unrelated,Zero\}$
\newline\textsc{After:}
\newline.~~~~ReceiveStatus=Blocked
\newline.~~~~SendStatus=Ok
\newline.~~~~SenderPreemption=No

\subsubsection{Transition 14}


\textsc{Before:}
\newline.~~~~$Id=Task$
\newline.~~~~$ReceiverState=Other$
\newline.~~~~$Satisfy=all$
\newline.~~~~$Send \in \{All,MixedAll\}$
\newline\textsc{After:}
\newline.~~~~ReceiveStatus=Satisfied
\newline.~~~~SendStatus=Ok
\newline.~~~~SenderPreemption=No

\subsubsection{Transition 15}


\textsc{Before:}
\newline.~~~~$Id=Task$
\newline.~~~~$ReceiverState=Other$
\newline.~~~~$Satisfy=Any$
\newline.~~~~$Send \in \{Any,MixedAny\}$
\newline\textsc{After:}
\newline.~~~~ReceiveStatus=Satisfied
\newline.~~~~SendStatus=Ok
\newline.~~~~SenderPreemption=No

\subsubsection{Transition 16}


\textsc{Before:}
\newline.~~~~$Id=Task$
\newline.~~~~$ReceiverState=Other$
\newline.~~~~$Satisfy=All$
\newline.~~~~$Send \in \{Any,MixedAny\}$
\newline\textsc{After:}
\newline.~~~~ReceiveStatus=Blocked
\newline.~~~~SendStatus=Ok
\newline.~~~~SenderPreemption=No
